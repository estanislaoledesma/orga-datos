\documentclass[a4paper,10pt]{article}
\usepackage{pdfpages}
\usepackage{listings}
\usepackage{graphicx}
\usepackage[ansinew]{inputenc}
\usepackage[spanish]{babel}


\title{		\textbf{TP 0: Set de Julia}}

\author{	Martín Queija, \textit{Padrón Nro. 96.455}                     \\
            \texttt{ tinqueija@gmail.com }                                              \\[2.5ex]
            Estanislao Ledesma, \textit{Padrón Nro. 96.622}                     \\
            \texttt{ estanislaoledesma@gmail.com }                                              \\[2.5ex]
            Agustin Luques, \textit{Padrón Nro. 96.803}                     \\
            \texttt{ agus.luques@hotmail.com }                                              \\[2.5ex]
            \normalsize{2do. Cuatrimestre de 2016}                                      \\
            \normalsize{66.20 Organización de Computadoras  $-$ Práctica Martes}  \\
            \normalsize{Facultad de Ingeniería, Universidad de Buenos Aires}            \\
       }

\begin{document}

\maketitle
\thispagestyle{empty}   % quita el número en la primer página


\begin{abstract}
\centerline{Graficador de Conjuntos de Julia en formáto PGM}

\end{abstract}
\newpage

\tableofcontents

\includepdf[pages={1,2,3,4}]{tp0-2016-2q.pdf}

\section{Introducción}

Este trabajo práctico tiene como objetivo realizar un graficador de Conjuntos de Julia. Este tipo de fractal se obtiene iterando distintos puntos especificos del plano complejo sobre una función holomorfa. La intención es realizar un sólido programa de linea de comando y que a su vez posea una especifica validación de parámetros. En caso de error deberia indicar al usuario especificamente en que parámetro se equivocó para facilitar su solución. Finalmente el programa debe cumplir con los estandarés de imagenes PGM en escala de grises. Detalles relacionados a PGM estan indicados en el enunciado del TP.


\section{Diseño e implementación}

Antes que nada, cabe informar que se utilizó una estructura llamada “complejo” para poder representarlos. La misma consiste de dos atributos de tipo float los cuales representan a la parte real e imaginaria del numero. El programa consiste primeramente en encontrar cada argumento indicado (pasado como parámetro al programa por línea de comandos), analizar si éstos son correctos, guardarlos en sus respectivas variables y, en caso de no ser correctos, informar el error y cortar la ejecución del programa.Todos los parámetros fueron guardados en variables estáticas, y en caso de no estar presentes se reemplazaron por aquellos dados en el enunciado por defecto. En caso de no poder crear un archivo de salida, también se sale del programa. Se escribió un programa secuencial completamente, sin el agregado de muchas estructuras adicionales ni funciones auxiliares, para que luego el seguimiento en lenguaje ensamblador sea más simple y entendible. El porgrama comienza con un ciclo switch en el que se almacenan todos los parámetros pasados al programa mediante línea de comandos, a través de la función getopt de la librería unistd. En caso de error, se sale del programa con EXIT-FAILURE. Para la verificación de la validez de los mismos se utilizaron distintas funciones auxiliares, como atoi para la conversión de string a entero, o sscanf para los parámetros que combinaban decimales y otros símbolos (como los complejos). En caso de error, se imprimen mensajes de error y se sale del programa con EXIT-FAILURE.La parte central del algoritmo, donde se itera en el plano complejo de forma discreta, se utilizan dos ciclos definidos for para el eje real y el eje imaginario, manteniendo un acumulador para definir el brillo del pixel, el cual luego es guardado en el archivo de salida. Para enontrar el paso de avance sobre los ejes del plano, se utilizó el cociente entre el alto o ancho y la resolución de la imagen, para el eje imaginario y real, respectivamente. La operación matemática que se realiza en el interior de los ciclos se implementó de la forma más eficiente posible encontrada, generando una única línea de código para la parte real e imaginaria, y usando una única variable temporal.

\section{Como obtener el ejecutable y correr pruebas}

Para mayor comodidad se dispone de un archivo llamado “script.sh” que al ejecutarlo (esto es, posicionándose sobre el directorio donde se encuentran los archivos y ejecutando el comando ./script.sh) automáticamente compilara el programa y ejecutará una serie de comandos de prueba y sus respectivas salidas.

\section{Código}
\begin{lstlisting}
#include <string.h>
#include <stdlib.h>
#include <stdio.h>
#include <ctype.h>
#include <math.h>
#include <unistd.h>
#include <getopt.h>
#define N 255

typedef struct complejo{
float parteReal;
float parteImaginaria;
}complejo;

void print_usage() {
printf("Error de parametros ./tp0 -h para ayuda detallada.\nUso: ./tp0 -r 640x480 -c 0+0i -C 0.285-0,01i -w 4 -H 4 -o /home/user/julia.pgm (or -o - for stdout)\n");
}


int main(int argc, char * const argv[]){
char* pathToExe = argv[0];
int option = -1;
int encontroRes = -1;
int encontroAlto = -1;
int encontroAncho = -1;
int encontroConst = -1;
int encontroOutput = -1;
int encontroOrigen = -1;
int salidaConsola = -1;
char* ptrOrigen;
char* constante;
char* output;
char* res;
char controli;
complejo constanteC;
complejo origen;
complejo pixel;
int resolucionHorizontal, resolucionVertical;
float anchoRectangulo;
float altoRectangulo;

while ((option = getopt(argc, argv,"r:c:C:W:H:o:h")) != -1) {
switch (option) {
case 'r' : 
encontroRes = 1;
res = optarg;
break;
case 'c' : 
encontroOrigen = 1;
ptrOrigen = optarg;
break;
case 'C' : 
encontroConst = 1;
constante = optarg; 
break;
case 'W' : 
anchoRectangulo = atoi(optarg);
if (anchoRectangulo <= 0){
printf("Ancho plano complejo invalido.\n");
exit(EXIT_FAILURE);
}
encontroAncho = 1;
break;
case 'H' : 
altoRectangulo = atoi(optarg);
if (altoRectangulo <= 0){
printf("Alto plano complejo invalido.\n");
exit(EXIT_FAILURE);
}
encontroAlto = 1;
break;
case 'o' : 
encontroOutput = 1;
if (strcmp(optarg,"-") == 0) salidaConsola = 1;
output = optarg;
/*
if ((isalpha((char)optarg[0]) != 1) && (isdigit((char)optarg[0]) != 1)){
if (strcmp(optarg, "-") == 0) salidaConsola = 1;
else {
printf("fatal: cannot open output file.\n");
exit(0);
}
}
output = optarg*/
break;
case 'h' : 
printf("AYUDA:	Resolucion: -r 640x480\n	Origen: -c 0+0i\n	Constante imaginaria: -C 0.285-0,01i\n	Ancho del plano complejo: -W 4\n	Alto del plano complejo: -H 4\n	Formato de salida: -o /home/julia.pgm (PARAMETRO IMPRESCINDIBLE) (or -o - for stdout)\n");
return 0;
default: 
print_usage(); 
exit(EXIT_FAILURE);
}
}

if (encontroOutput != 1){   ///UNICA CONDICION NECESARIA PARA CORRER EL PROGRAMA
printf("El argumento -o es condición necesaria\n");
exit(EXIT_FAILURE);
} 

if (encontroAlto != 1)	altoRectangulo = 4;
if (encontroAncho != 1)	anchoRectangulo = 4;

int scanResult;
if (encontroConst == 1){
scanResult = sscanf(constante, "%f%f%c", &constanteC.parteReal, &constanteC.parteImaginaria, &controli);
if (scanResult < 3){
printf("Error obteniendo constante compleja.\n");
print_usage();
exit(EXIT_FAILURE);
}
} else {
constanteC.parteReal = 0.285;
constanteC.parteImaginaria = -0.01;
}

if (encontroOrigen == 1){
scanResult = sscanf(ptrOrigen, "%f%f%c", &origen.parteReal, &origen.parteImaginaria, &controli);
if (scanResult < 3 || controli != 'i'){
printf("Error obteniendo origen complejo.\n");
print_usage();
exit(EXIT_FAILURE);
}
} else {
origen.parteReal = 0;
origen.parteImaginaria = 0;	
}

if (encontroRes == 1){
int scanResult = sscanf(res, "%d%c%d", &resolucionHorizontal, &controli, &resolucionVertical);
if (scanResult < 2 || resolucionVertical <= 0 || resolucionHorizontal <= 0 || controli != 'x')
{
if (resolucionVertical <= 0) printf("Resolucion vertical invalida\n");
else if (resolucionHorizontal <= 0) printf("Resolucion horizontal invalida\n");
else printf("Resolucion invalida\n");
print_usage();
exit(EXIT_FAILURE);
}
} else {
resolucionHorizontal = 640;
resolucionVertical = 480;
}


float pasoHorizontal = ((anchoRectangulo) / (float)(resolucionHorizontal))/2;
float pasoVertical = ((altoRectangulo) / (float)(resolucionVertical))/2;
int contadorBrillo;
float temp, valorAbsoluto;

FILE *fp;
fprintf(stdout, "P2 \n%d %d \n255 \n", resolucionHorizontal, resolucionVertical);	
if (salidaConsola != 1) {
fp=fopen(output, "w");
if(fp == NULL) {
printf("fatal: cannot open output file\n");
print_usage();
exit(EXIT_FAILURE);
}
fprintf(fp, "P2 \n%d %d \n255 \n", resolucionHorizontal, resolucionVertical);	
}

//debug printf("constantR %f constanteI %f orinenR %f origenI %f \n", constanteC.parteReal, constanteC.parteImaginaria, origen.parteReal, origen.parteImaginaria);	

for (int y = 1; y <= resolucionVertical; y++){
for (int x = 1; x <= resolucionHorizontal; x++){
pixel.parteReal = pasoHorizontal * (2 * x -1) - anchoRectangulo / 2 + origen.parteReal;
pixel.parteImaginaria = -pasoVertical * (2 * y -1) + altoRectangulo / 2 + origen.parteImaginaria;

for (contadorBrillo = 0; contadorBrillo <= N; contadorBrillo++){
valorAbsoluto = sqrtf((pixel.parteReal * pixel.parteReal) + (pixel.parteImaginaria * pixel.parteImaginaria));
if (valorAbsoluto > 2) break;
temp = ((pixel.parteReal)*(pixel.parteReal))-((pixel.parteImaginaria)*(pixel.parteImaginaria)) + constanteC.parteReal;
pixel.parteImaginaria = ((pixel.parteImaginaria)*(pixel.parteReal))+((pixel.parteReal)*(pixel.parteImaginaria)) + constanteC.parteImaginaria;
pixel.parteReal = temp;
}
if (salidaConsola == 1) fprintf(stdout, "%d ", contadorBrillo); else fprintf(fp, "%d ", contadorBrillo );
}
if (salidaConsola == 1) fprintf(stdout, "\n"); else fprintf(fp, "\n");
}

if (salidaConsola != 1){
rewind(fp);
fclose(fp);
printf("Archivo guardado en %s/%s\n", pathToExe, output);
}
/*
if (salidaConsola == 1){

fp=fopen(output, "rb+");
if(fp == NULL) return 1;
int c;

while((c=fgetc(fp)) != EOF){
printf( "%c", c );
}

fclose(fp);
remove(output);
}*/
return 0;
}

\end{lstlisting}

\section{Código MIPS (Primeras hojas)}
\begin{lstlisting}
.section .mdebug.abi32
	.previous
	.abicalls
	.file 1 "tp0f.c"
	.section	.debug_abbrev,"",@progbits
$Ldebug_abbrev0:
	.section	.debug_info,"",@progbits
$Ldebug_info0:
	.section	.debug_line,"",@progbits
$Ldebug_line0:
	.text
$Ltext0:
	.file 2 "/usr/include/mips/int_types.h"
	.file 3 "/usr/include/mips/ansi.h"
	.file 4 "/usr/include/string.h"
	.file 5 "/usr/include/mips/types.h"
	.file 6 "/usr/include/sys/ansi.h"
	.file 7 "/usr/include/sys/types.h"
	.file 8 "/usr/include/sys/endian.h"
	.file 9 "/usr/include/pthread_types.h"
	.file 10 "/usr/include/stdlib.h"
	.file 11 "/usr/include/stdio.h"
	.file 12 "/usr/include/math.h"
	.file 13 "/usr/include/unistd.h"
	.file 14 "/usr/include/getopt.h"
	.rdata
	.align	2
$LC0:
	.ascii	"Error de parametros ./tp0 -h para ayuda detallada.\n"
	.ascii	"Uso: ./tp0 -r 640x480 -c 0+0i -C 0.285-0,01i -w 4 -H 4 -"
	.ascii	"o /home/user/julia.pgm (or -o - for stdout)\n\000"
	.text
	.align	2
	.globl	print_usage
$LFB29:
	.loc 1 15 0
	.ent	print_usage
print_usage:
	.frame	$fp,40,$31		# vars= 0, regs= 3/0, args= 16, extra= 8
	.mask	0xd0000000,-8
	.fmask	0x00000000,0
	.set	noreorder
	.cpload	$25
	.set	reorder
	subu	$sp,$sp,40
	.cprestore 16
$LCFI0:
	sw	$31,32($sp)
$LCFI1:
	sw	$fp,28($sp)
$LCFI2:
	sw	$28,24($sp)
$LCFI3:
	move	$fp,$sp
$LCFI4:
	.loc 1 16 0
	la	$4,$LC0
	la	$25,printf
	jal	$31,$25
	.loc 1 17 0
	move	$sp,$fp
	lw	$31,32($sp)
	lw	$fp,28($sp)
	addu	$sp,$sp,40
	j	$31
	.end	print_usage
$LFE29:
	.size	print_usage, .-print_usage
	.rdata
	.align	2
$LC1:
	.ascii	"r:c:C:W:H:o:h\000"
	.align	2
$LC2:
	.ascii	"Ancho plano complejo invalido.\n\000"
	.align	2
$LC3:
	.ascii	"Alto plano complejo invalido.\n\000"
	.align	2
$LC4:
	.ascii	"-\000"
	.align	2
$LC5:
	.ascii	"AYUDA:\tResolucion: -r 640x480\n"
	.ascii	"\tOrigen: -c 0+0i\n"
	.ascii	"\tConstante imaginaria: -C 0.285-0,01i\n"
	.ascii	"\tAncho del plano complejo: -W 4\n"
	.ascii	"\tAlto del plano complejo: -H 4\n"
	.ascii	"\tFormato de salida: -o /home/julia.pgm (PARAMETRO IMPRE"
	.ascii	"SCINDIBLE) (or -o - for stdout)\n\000"
	.align	2
$LC6:
	.ascii	"El argumento -o es condici\303\263n necesaria\n\000"
	.align	2
$LC8:
	.ascii	"%f%f%c\000"
	.align	2
$LC9:
	.ascii	"Error obteniendo constante compleja.\n\000"
	.align	2
$LC12:
	.ascii	"Error obteniendo origen complejo.\n\000"
	.align	2
$LC13:
	.ascii	"%dx%d\000"
	.align	2
$LC14:
	.ascii	"Resolucion vertical invalida\n\000"
	.align	2
$LC15:
	.ascii	"Resolucion horizontal invalida\n\000"
	.align	2
$LC17:
	.ascii	"w\000"
	.align	2
$LC18:
	.ascii	"fatal: cannot open output file\n\000"
	.align	2
$LC19:
	.ascii	"P2 \n"
	.ascii	"%d %d \n"
	.ascii	"255 \n\000"
	.align	2
$LC20:
	.ascii	"%d \000"
	.align	2
$LC21:
	.ascii	"\n\000"
	.align	2
$LC22:
	.ascii	"Archivo guardado en %s/%s\n\000"
	.align	2
$LC7:
	.word	1082130432
	.align	2
$LC10:
	.word	1049750405
	.align	2
$LC11:
	.word	-1138501878
	.align	2
$LC16:
	.word	1073741824
	.text
	.align	2
	.globl	main
$LFB31:
	.loc 1 20 0
	.ent	main
main:
	.frame	$fp,192,$31		# vars= 144, regs= 3/0, args= 24, extra= 8
	.mask	0xd0000000,-8
	.fmask	0x00000000,0
	.set	noreorder
	.cpload	$25
	.set	reorder
	subu	$sp,$sp,192
	.cprestore 24
$LCFI5:
	sw	$31,184($sp)
$LCFI6:
	sw	$fp,180($sp)
$LCFI7:
	sw	$28,176($sp)
$LCFI8:
	move	$fp,$sp
$LCFI9:
	sw	$4,192($fp)
	sw	$5,196($fp)
	.loc 1 21 0
$LBB2:
	lw	$2,196($fp)
	lw	$2,0($2)
	sw	$2,32($fp)
	.loc 1 22 0
	li	$2,-1			# 0xffffffffffffffff
	sw	$2,36($fp)
	.loc 1 23 0
	li	$2,-1			# 0xffffffffffffffff
	sw	$2,40($fp)
	.loc 1 24 0
	li	$2,-1			# 0xffffffffffffffff
	sw	$2,44($fp)
	.loc 1 25 0
	li	$2,-1			# 0xffffffffffffffff
	sw	$2,48($fp)
	.loc 1 26 0
	li	$2,-1			# 0xffffffffffffffff
	sw	$2,52($fp)
	.loc 1 27 0
	li	$2,-1			# 0xffffffffffffffff
	sw	$2,56($fp)
	.loc 1 28 0
	li	$2,-1			# 0xffffffffffffffff
	sw	$2,60($fp)
	.loc 1 29 0
	li	$2,-1			# 0xffffffffffffffff
	sw	$2,64($fp)
	.loc 1 42 0
$L19:
	lw	$4,192($fp)
	lw	$5,196($fp)
	la	$6,$LC1
	la	$25,getopt
	jal	$31,$25
	sw	$2,36($fp)
	lw	$3,36($fp)
	li	$2,-1			# 0xffffffffffffffff
	bne	$3,$2,$L21
	b	$L20
$L21:
	.loc 1 43 0
	lw	$2,36($fp)
	addu	$2,$2,-67
	sw	$2,168($fp)
	lw	$3,168($fp)
	sltu	$2,$3,48
	beq	$2,$0,$L35
	lw	$2,168($fp)
	sll	$3,$2,2
	la	$2,$L36
	addu	$2,$3,$2
	lw	$2,0($2)
	.cpadd	$2
	j	$2
	.rdata
	.align	2
\end{lstlisting}
\section{Conclusiones}
Se tarda poco en realizarse que el verdadero desafío que se presenta durante la codificación del programa es la validación de los párametros. El algorítmo para obtener el set de Julia es fácil de implementar y no debería de requerir funciones adicionales. Resultó interesante incrementar la resolución para la misma superficie compleja y observar cuan profundo se puede \textit{zoomear} en las imagenes resultantes.
\begin{thebibliography}{99}
\bibitem{INT06} Conjunto de Julia, WikiPedia, https://es.wikipedia.org/wiki/Conjunto\_de\_Julia.
\bibitem{INT06} The C Programming Language by Brian W. Kernighan and Dennis M. Ritchie.
\end{thebibliography}
\end{document}