\documentclass[a4paper,10pt]{article}
\usepackage{pdfpages}
\usepackage{listings}
\usepackage{graphicx}
\usepackage[ansinew]{inputenc}
\usepackage[spanish]{babel}

\title{		\textbf{Diseño del TP: Fine Foods Review}}

\author{	Martín Queija, \textit{Padrón Nro. 96.455}                     \\
            \texttt{ tinqueija@gmail.com }                                              \\[2.5ex]
            Estanislao Ledesma, \textit{Padrón Nro. 96.622}                     \\
            \texttt{ estanislaoledesma@gmail.com }                                              \\[2.5ex]
            Martín Bosch, \textit{Padrón Nro. 00.000}                     \\
            \texttt{ agus.luques@hotmail.com }                                              \\[2.5ex]
            \normalsize{2do. Cuatrimestre de 2016}                                      \\
            \normalsize{Organización de Datos  }  \\
            \normalsize{Facultad de Ingeniería, Universidad de Buenos Aires}            \\
       }
\date{}

\begin{document}

\maketitle
\thispagestyle{empty}   % quita el número en la primer página


\begin{abstract}
\centerline{Aproximación por regresión}

\end{abstract}
\newpage

\tableofcontents


\section{Introducción}

Aca va la introducción

\section{Diseño}

Aca va el diseño

\section{Otro titulo que podamos poner}

mas texto

\section{Conclusiones}

Nuestra conclusion

\begin{thebibliography}{99}

\bibitem{INT06} Conjunto de Julia, WikiPedia, https://es.wikipedia.org/wiki/Conjunto\_de\_Julia.
\bibitem{INT06} The C Programming Language by Brian W. Kernighan and Dennis M. Ritchie.

\end{thebibliography}

\end{document}
