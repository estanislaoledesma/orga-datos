\documentclass[a4paper,10pt]{article}
\usepackage{pdfpages}
\usepackage{listings}
\usepackage{graphicx}
\usepackage[ansinew]{inputenc}
\usepackage[spanish]{babel}

\title{		\textbf{Diseño del TP: Fine Foods Review}}

\author{	Martín Queija, \textit{Padrón Nro. 96.455}                     \\
	\texttt{ tinqueija@gmail.com }                                              \\[2.5ex]
	Estanislao Ledesma, \textit{Padrón Nro. 96.622}                     \\
	\texttt{ estanislaoledesma@gmail.com }                                              \\[2.5ex]
	Martín Bosch, \textit{Padrón Nro. 00.000}                     \\
	\texttt{ agus.luques@hotmail.com }                                              \\[2.5ex]
	\normalsize{2do. Cuatrimestre de 2016}                                      \\
	\normalsize{Organización de Datos  }  \\
	\normalsize{Facultad de Ingeniería, Universidad de Buenos Aires}            \\
}
\date{}

\begin{document}
	
	\maketitle
	\thispagestyle{empty}   % quita el número en la primer página
	
	
	\begin{abstract}
		\centerline{Aproximación por regresión}
		
	\end{abstract}
	\newpage
	
	\tableofcontents
	
	
	\section{Introducción}
	
	El objetivo de este trabajo práctico es predecir la puntuación de distintas reseñas de comidas.
	
	\section{Diseño}
	
	Para alcanzar el objetivo, la intención es diseñar un algoritmo que \textit{aprenda} a partir de un set de reseñas ya puntuadas para así poder predecir la calificación de nuevas reseñas. Parte del set de reseñas ya puntuadas se reserva para probar el "aprendisaje" del algorítmo. De esta manera se puede determinar la presicion del algorítmo al predecir reseñas y comparar la predicción con el valor esperado. La comparacion utilizada se denomina Mean Squared Error Esta relación proporciona un grado de confianza sobre nuestro algorítmo para predecir puntuaciones de reseñas.
	
	
	\subsection{KNN}
	La base del diseño se basa en el método de reconocimiento de patrones llamado KNN (K-Nearest-Neighbour). Este método es aplicable para realizar regresión o clasificación. El set de datos provisto califica las reseñas con enteros del 1 al 5. Sin embargo el enunciado afirma que nuestro algorítmo puede predecir valores no enteros en el intervalo [1,5]. Por lo tanto establecemos que nuestro método de predicción sera regresivo.
	
	El mecanismo de KNN consiste en encontrar los K-Vecinos mas cercanos a la nueva instancia de datos que queremos clasificar. Para clasificación se predice la clase mayoritaria entre los K-Vecinos. Para regresión se calcula el promedio.
	
	El hyperparámetro K se deduce a partir de prueba y error. Probaremos diferentes valores de K para encontrar el que mejor ajuste a nuestro set de datos.
	
	\subsection{Apache Spark}
	El manejo de datos será responsabilidad de Spark. Apache Spark es un framework para implementar \textit{cluster computing.} Spark nos permite manipular el set de datos paralelamente de manera eficiente. 
	
	
	\section{Otro titulo que podamos poner}
	
	mas texto
	
	\section{Conclusiones}
	
	Nuestra conclusion
	
	\begin{thebibliography}{99}
		
		\bibitem{INT06} Conjunto de Julia, WikiPedia, https://es.wikipedia.org/wiki/Conjunto\_de\_Julia.
		\bibitem{INT06} The C Programming Language by Brian W. Kernighan and Dennis M. Ritchie.
		
	\end{thebibliography}
	
\end{document}
